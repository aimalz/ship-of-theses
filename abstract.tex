The study of exoplanets has been revolutionized in recent years thanks, in
large part, to new data collected by NASA's \emph{Kepler} Mission.
The Mission has enabled the discovery of thousands of planets orbiting stars
throughout the Galaxy.
These discoveries span orders of magnitude in physical parameter space but
many of the most physically interesting questions remain open.
The deepest of these questions is: how common are planetary systems like our
own Solar System?
In this dissertation, I approach this question from several different angles
and make inferences about the frequency and distribution of planets based on
the large, publicly-available datasets from the \emph{Kepler} Mission.

I develop two powerful and practical methods for mining for planetary transit
signals in the hundreds of thousands of stellar light curves measured by
\emph{Kepler}.
The first method is designed to find planets using the data from the
\emph{K2} phase of the Mission where systematics introduced by the instrument
dominate the measurements.
Applying this method to the first publicly available dataset from \emph{K2},
Campaign 1, I published more than thirty new exoplanet candidates.
The second transit search technique is designed to find transits of planets
with orbital periods longer than the four year baseline of the \emph{Kepler}
Mission.
These are interesting planets because they are expected to have the largest
dynamical influence on the formation and evolution of their planetary systems
but, to date, no systematic search for these signals has been published.
I demonstrate that this method is robust and tractable and make predictions
for the planet yields in the \emph{Kepler} dataset.

I derive a general framework for making justified probabilistic inferences
about the population of planets based on noisy and incomplete catalogs of
exoplanet measurements.
Applying this to a previously published catalog of exoplanets orbiting stars
like our Sun, I measure the joint period--radius distribution of these
planets taking into account survey selection effects and the large
measurement uncertainties.
Despite the fact that this catalog includes no true Earth analogs, I use the
detected systems and weak smoothness assumptions about the underlying
distribution to make a probabilistic estimate of the frequency of Earth-like
planets.
