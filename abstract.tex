Upcoming optical astronomical surveys, including the Large Synoptic Survey Telescope (LSST), will produce tremendous galaxy catalogs from purely photometric data.
Spectroscopic redshift confirmation will be impossible not only because of the sheer number of galaxies, but also because of the unprecedented depth of the data, which will include a substantial population of faint, low signal-to-noise galaxies.
Existing analysis techniques in cosmology and population-level studies of galaxies make assumptions regarding the nature of the data that will not hold in this new regime, rendering traditional statistical techniques invalid.
This thesis develops novel methodologies for cosmology using probabilistic descriptions appropriate for the uncertainty-dominated data anticipated of next-generation observational astrophysics missions. 

I develop a statistically rigorous approach to a cosmological application of probabilistic photometric redshifts.
I execute original proofs covering the only circumstances under which a ubiquitous heuristic yields an accurate estimator of the redshift distribution from a sample of redshift probabilities.
I introduce Cosmological Hierarchical Inference with Probabilistic Photometric Redshifts (CHIPPR), a hierarchical Bayesian model and accompanying public code that enables the use of an existing catalog of redshift posteriors in a mathematically self-consistent inference of the redshift distribution that appears in cosmological analyses.
I compare the performance of CHIPPR to familiar alternatives using instructive cases of forward-modeled mock data, from toy to realistically complex.

I present a comprehensive assessment of twelve approaches to deriving photometric redshift probabilities under controlled experimental conditions in the context of LSST.
I demonstrate the impact of the assumptions implicit to the method by which the redshift probabilities are obtained, and I identify the limitations of established performance metrics of such probabilistic data products.
I also address the question of how enormous sets of probabilistic data products such as redshift probabilities should be delivered to those aiming to use them in generic physical inference, balancing constraints on available storage resources and requirements on preserved information content.


