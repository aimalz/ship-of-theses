\RequirePackage[displaymath, mathlines]{lineno}

\usepackage[tbtags]{amsmath}
\usepackage{amssymb}
\usepackage{amsfonts}
\usepackage{amsthm}

\usepackage{color, hyperref}
\definecolor{linkcolor}{rgb}{0,0,0.2}
\hypersetup{colorlinks=true,linkcolor=linkcolor,citecolor=linkcolor,
            filecolor=linkcolor,urlcolor=linkcolor}
\hypersetup{pageanchor=false}

\usepackage{indentfirst}
\usepackage{booktabs}
\usepackage{url}
\usepackage{multirow}
\usepackage{xspace}
\usepackage{enumitem}
\usepackage[final]{graphicx}
\usepackage{wasysym}

\usepackage{footmisc}
\usepackage{soul}

\usepackage{tikz}
\usetikzlibrary{shapes.geometric, arrows}
\usetikzlibrary{fit}

\tikzstyle{hyper} = [circle, text centered, draw=black]
\tikzstyle{param} = [circle, text centered, draw=black]
\tikzstyle{data} = [circle, text centered, draw=black, line width=2pt]
\tikzstyle{arrow} = [thick,->,>=stealth]

% Custom AAS macros:
\usepackage{aas_macros}
\usepackage{lsstdesc_macros}

%% Just for committee review!  TODO: comment out this entire block before submitting the real thing
%\usepackage{fancyhdr}
%\usepackage{datetime}
%\fancyhf{}
%\fancyhead[R]{\today\ \currenttime}
%\pagestyle{fancy}

% Bibliography:
\usepackage{natbib}
\bibliographystyle{apj}

% Algorithms:
\usepackage{algorithmic, algorithm}

% Thesis-specific:
\newcommand{\thesistitle}{Probabilistic analysis methods for cosmology \\
	using uncertainty-dominated photometric data}
\newcommand{\thesisauthor}{Alex I. Malz}
\newcommand{\thesisadvisor}{Professor David W. Hogg}
\newcommand{\graddate}{May 2019}

% Commands:

% General formatting:
\newcommand{\paper}{Chapter}
%\newcommand{\textul}[1]{{\underline{#1}}}
\newcommand{\mathul}[1]{\underline{#1}}
\newcommand{\enquote}[1]{{``{#1}''}}

% Foreign text:
\newcommand{\foreign}[1]{\emph{#1}}
%\newcommand{\etal}{\foreign{et\,al.}}
%\newcommand{\etc}{\foreign{etc.}}

% Project references:
\newcommand{\project}[1]{{\textsc{#1}}}
\newcommand{\lsst}{\project{LSST}}
\newcommand{\desc}{\project{LSST-DESC}}
\newcommand{\coin}{\project{COIN}}
\newcommand{\sdss}{\project{SDSS}}
\newcommand{\des}{\project{DES}}

% Code references:
\newcommand{\repo}[1]{{\texttt{#1}}}
\newcommand{\qp}{\repo{qp}}
\newcommand{\chippr}{\repo{chippr}}

% Tools:
\newcommand{\github}{\href{https://github.com}{GitHub}}
\newcommand{\python}{\textit{Python}}

% LaTeX object referencing:
\newcommand{\chapid}{no chapter}
\renewcommand{\figref}[1]{\ref{\chapid:fig:#1}}
\newcommand{\Fig}[1]{Figure~\figref{#1}}
\newcommand{\fig}[1]{\Fig{#1}}
\newcommand{\figlabel}[1]{\label{\chapid:fig:#1}}

\newcommand{\Tab}[1]{Table~\ref{\chapid:tab:#1}}
\newcommand{\tab}[1]{\Tab{#1}}
\newcommand{\tablabel}[1]{\label{\chapid:tab:#1}}

\renewcommand{\eqref}[1]{\ref{\chapid:eq:#1}}
\newcommand{\Eq}[1]{Equation~(\eqref{#1})}
\newcommand{\eq}[1]{\Eq{#1}}
\newcommand{\eqalt}[1]{Equation~\eqref{#1}}
\newcommand{\eqlabel}[1]{\label{\chapid:eq:#1}}

\newcommand{\sectionname}{Section}
\newcommand{\sectref}[1]{\ref{\chapid:sect:#1}}
\newcommand{\Sect}[1]{\sectionname~\sectref{#1}}
\newcommand{\sect}[1]{\Sect{#1}}
\newcommand{\sectalt}[1]{\sectref{#1}}
\newcommand{\App}[1]{Appendix~\sectref{#1}}
\newcommand{\app}[1]{\App{#1}}
\newcommand{\sectlabel}[1]{\label{\chapid:sect:#1}}

\newcommand{\Algo}[1]{Algorithm~\ref{\chapid:algo:#1}}
\newcommand{\algo}[1]{\Algo{#1}}
\newcommand{\algolabel}[1]{\label{\chapid:algo:#1}}

\newcommand{\chapname}{Chapter}
\newcommand{\Chap}[1]{\chapname~\ref{chap:#1}}
\newcommand{\chap}[1]{\Chap{#1}}
\newcommand{\chapalt}[1]{\ref{chap:#1}}
\newcommand{\chaplabel}[1]{\label{chap:#1}}

\newcommand{\todo}[3]{{\color{#2}\emph{#1}: #3}}
\newcommand{\aim}[1]{\todo{AIM}{red}{#1}}

\newtheorem{theorem}{Theorem}[section]
\newtheorem{proposition}[theorem]{Proposition}
\newtheorem{lemma}[theorem]{Lemma}
\newtheorem{corollary}[theorem]{Corollary}
\newtheorem{conjecture}[theorem]{Conjecture}
\theoremstyle{definition}
\newtheorem{definition}[theorem]{Definition}
\newtheorem{remark}[theorem]{Remark}
\newtheorem{example}[theorem]{Example}

% Math:
%\newcommand{\dd}{\ensuremath{\,\mathrm{d}}}
\newcommand{\bvec}[1]{{\ensuremath{\boldsymbol{#1}}}}% could change to \vec
%\newcommand{\paramvector}[1]{\bvec{#1}}
%\newcommand{\unit}[1]{\mathrm{#1}}
\newcommand{\data}{\ensuremath{\vec{d}}}% could change to bold

% Probabilities:
\newcommand{\perm}[2]{\ensuremath{_{#1}\mathrm{P}_{#2}}}
\newcommand{\comb}[2]{\ensuremath{_{#1}\mathrm{C}_{#2}}}
\newcommand{\like}{\mathscr{L}}
\newcommand{\pr}[1]{\ensuremath{\mathrm{p}(#1)}}% could change to Prob or Pr 
\newcommand{\expect}[1]{\left<#1\right>}
\newcommand{\normal}[2]{\mathcal{N} (#1, #2)}
\newcommand{\gvn}{\mid}% could use | or \vert
\newcommand{\integral}[2]{\ensuremath{\int\ #1\ \mathrm{d} #2}}

% misc science
\newcommand{\sz}{spec-$z$}
\newcommand{\Sz}{Spec-$z$}
\newcommand{\pz}{photo-$z$}
\newcommand{\Pz}{Photo-$z$}
\newcommand{\pzpdf}{\pz\ PDF}% could change to posterior
\newcommand{\Pzpdf}{\Pz\ PDF}% could change to posterior
\newcommand{\zpdf}{redshift posterior}% could change to implicit posterior
\newcommand{\pzip}{\pz\ implicit posterior}
\newcommand{\nz}{$n(z)$}
\newcommand{\Nz}{$N(z)$}
\newcommand{\stack}{$\hat{N}(z)$}

\sloppy\sloppypar
