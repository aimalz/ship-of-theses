I'll start by thanking my advisor, David Hogg, whose passion for astronomical data has proven as contagious as his approach to scientific discourse.
By funding a free-range grad student, Hogg gave me the opportunity to teach myself to forge new collaborations, the skill that has made possible this dissertation, as well as most of the works in progress that didn't make the cut.
If Hogg taught me how to start projects, it's Phil Marshall who taught me how to finish them.
By taking me seriously before I had any legitimacy and supporting my efforts in the larger context of \desc, Phil shaped me into the functional researcher I am today.
Thank you both for catalyzing crazy idea after crazy idea, for making wager after wager to encourage me to follow through, and for never letting me forget the real reason we do this \textemdash because it's \textit{fun}!

I'm grateful to the entire \desc, for providing me with an endless supply of hopeless, and thus interesting, problems, and for supporting me in investigating solutions, however speculative they may seem. 
I'd especially like to thank Michael Schneider, Sam Schmidt, Jeff Newman, and Anja von der Linden on behalf of the Weak Lensing, Photo-z, and Clusters Working Groups, as well as the Supernova Working Group and the Photometric LSST Astronomical Time-series Classification Challenge (\project{PLAsTiCC}) team, for believing in me and amplifying my voice before I had the credibility to be heard on my own.

This list wouldn't be complete without an expression of gratitude to all the hacking hackers who taught me to hack, from Daniela Huppenkothen and the Astro Hack Week family to Emille Ishida and the Cosmostatistics Initiative, to whom I owe a great debt for equipping me with the tools, community, and confidence to approach any problem.

%	\item I thank the Flatiron Institute for generous nutritional support.
%	\item I thank Laura Kay and Steinn Sigurdsson for facilitating my transfer to NYU -- I wouldn't be getting a Ph.D. without you.
%	\item I express gratitude to Habitica, the gamified to-do list that motivated hundreds of hours of research.

This Ph.D. wouldn't have been nearly as fun nor productive without my peers at the NYU CCPP, especially MJ, Chang, Geoff, Kilian, Nitya, and David, who adopted me as one of their own and blurred the line between distraction and enrichment, on top of tolerating all my dumb questions over the years.
I'm immensely grateful to Asumana Randolph for decades of never giving up on me, and to all the Hunter College High School Freaks who welcomed me back to New York City with open arms.
Finally, I thank my spousal unit, Marty, who has stuck with me through over fifteen years of schooling and mooching and without whom I never would have become my dream.