\renewcommand{\kepler}{\project{Kepler}}
\newcommand{\KT}{\project{K2}}
\newcommand{\tess}{\project{TESS}}
\newcommand{\jwst}{\project{JWST}}
\newcommand{\pdc}{\project{PDC}}

\chapter{A systematic search for transiting planets in the \KT\ data}

\section{Chapter Abstract}

Photometry of stars from the \KT\ extension of NASA's \kepler\ mission is
afflicted by systematic effects caused by small (few-pixel) drifts in the
telescope pointing and other spacecraft issues.
We present a method for searching \KT\ light curves for evidence of
exoplanets by simultaneously fitting for these systematics and the
transit signals of interest.
This method is more computationally expensive than standard search algorithms
but we demonstrate that it can be efficiently implemented and used to
discover transit signals.
We apply this method to the full Campaign~1 dataset and report a list of 36
planet candidates transiting 31 stars, along with an analysis of the pipeline
performance and detection efficiency based on artificial signal injections
and recoveries.
For all planet candidates, we present posterior distributions on the
properties of each system based strictly on the transit observables.

\section{Introduction}\sectlabel{intro}

