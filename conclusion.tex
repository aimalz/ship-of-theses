\chapter*{Conclusion}\addcontentsline{toc}{chapter}{Conclusion}

In this dissertation, we study the population of exoplanets using data from
NASA's \kepler\ Mission and the re-purposed \KT\ Mission.
We develop and apply novel techniques to discover previously unknown planets
and planet candidates (\chapname s~\chapalt{ketu} and~\chapalt{peerless}).
We present a robust probabilistic framework for making inferences about the
population of exoplanets based on the noisy and incomplete catalogs derived
from transit surveys (\chap{exopop}).
The main contributions of this dissertation are methodological and each
\chapname\ is accompanied by open source software implementing the methods.

In the spirit of tool development and open source software, \Chap{emcee} is
describes \project{emcee}, a general purpose Markov Chain Monte Carlo sampler
that, since its release \citep{Foreman-Mackey:2013}, has become one of the
most popular tools for probabilistic inference in astronomy.
This method was originally proposed by \citet{Goodman:2010} and it was
designed to sample problems efficiently with little tuning even when the
parameter space is poorly conditioned.
This feature is especially useful for problems in astronomy where the physical
parameters often vary (and covary) over many orders of magnitude.
The \project{emcee} implementation offers a small performance gain by deriving
a parallelizable version of the original algorithm and a user-friendly and
well documented Python interface.
In practice, this method doesn't scale well to large numbers of dimensions
($\gtrsim 50$) but it has been shown to work out-of-the-box on a large class
of typical astronomy problems.

In \Chap{exopop}, we derive a hierarchical method for inferring the population
of exoplanets based on a catalog of planets with a non-trivial completeness
function and large measurement uncertainties.
This method builds on the importance sampling technique originally derived by
\citet{Hogg:2010a} to make a histogram of noisy measurements.
Applying this population inference method to a catalog of planet candidates
transiting Sun-like stars \citep{Petigura:2013}, we make a prediction for the
rate of Earth analogs.
This prediction is substantially lower than earlier predictions based on the
same catalog.
We demonstrate that this discrepancy is caused by both the treatment of the
observational uncertainties and the choice of extrapolation function.

In Summer 2014, the \kepler\ spacecraft was re-purposed and it began taking
data for the \KT\ Mission.
The pointing accuracy in this mode is substantially degraded relative to the
original Mission but, in \chap{ketu}, we demonstrate that these light curves
can still be used to systematically search for transiting exoplanets.
By building a flexible data-driven model for the systematic variability in the
light curves of the stars and combining this with an approximate linear
transit model, we derive a transit search algorithm where the systematics
model is marginalized for every hypothesis.
This enables the discovery of transit signals with amplitudes smaller than the
pointing-induced variability.
In \chap{ketu}, we announce the discovery of 36 planet candidates transiting
33 stars.
Of these candidates, 18 have been validated as bona fide planets and 6 have
been identified as likely astrophysical false positives
\citep{Crossfield:2015, Montet:2015, Armstrong:2015a}.

